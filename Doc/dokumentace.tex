\documentclass[12pt]{article}
\usepackage[utf8]{inputenc}
\usepackage[czech]{babel}
\usepackage[a4paper, top=2cm, bottom=2cm, left=2cm, right=2cm]{geometry}
\usepackage[IL2]{fontenc}
\usepackage{url}
\usepackage{indentfirst}
\usepackage{amsthm}
\usepackage{amsmath}
\usepackage{amsfonts}
\usepackage{graphicx}
\usepackage{caption}
\usepackage{subcaption}
\usepackage{color}
\usepackage{float}
\usepackage{multirow}
\usepackage{booktabs}
\usepackage[table,usenames,dvipsnames,svgnames]{xcolor}
\usepackage[unicode,hyperindex,plainpages=false,pdftex]{hyperref}
\usepackage{algorithm}
\usepackage[noend]{algpseudocode}   
    \hypersetup{
          colorlinks=true, 
          linkcolor=Red, 
          citecolor=Red, 
          filecolor=magenta, 
          urlcolor=TealBlue
    }

\newcommand{\todo}[1]{\textcolor{red}{[[TODO: #1]]}}
\newcommand{\setHelper}[1]{\left\lbrace #1 \right\rbrace}
\newlength{\skipeqarray}
\setlength{\skipeqarray}{0.4cm}




\title{Vyhledávání tRNA genů}
\author{Milan Munzar \\
\normalsize{\href{mailto:xmunza00@stud.fit.vutbr.cz}{xmunza00@stud.fit.vutbr.cz}}}
\date{}

\definecolor{lightgray}{rgb}{0.9, 0.9, 0.9}
\newtheorem{veta}{Věta}
\newtheorem{definition}{Definice}
\newtheorem{priklad}{Příklad}
\newfloat{algorithm}{h}{lop}
\floatname{algorithm}{Algoritmus}
\renewcommand{\algorithmiccomment}[1]{\hfill\# #1}
\algrenewcommand\alglinenumber[1]{{\footnotesize#1}}
    
\begin{document}
\maketitle

\section{Úvod}
Tento text popisuje návrh a implementaci algoritmu na vyhledávání tRNA genů uvnitř genomu. Program je napsán v jazyce Python 2.7 a tvoří ho dva skripty. Skript \texttt{get\_tRNA.py} nalezne tRNA geny a skript \texttt{compare\_tRNA.py} porovná nalezené geny s geny z Genomic tRNA Database(GTD). Program je testován na záznamech k bakterii Escherichia coli CFT073.

\section{Implementace vyhledávání tRNA genů}
Pro hledání tRNA genů jsem využil konzervanosti hledaných oblastí~\cite{saks}. Tato vlastnost dovoluje sestavit regulární výraz a aplikovat jej na celý genom v sense i anti-sense směru (alg:~\ref{re}). Hledání tRNA genů v anti-sense směru vyžaduje vytvoření komplementárního řetězce a jeho otočení. Výhoda této implementace je, že algoritmus pracuje s lineární časovou složitosti. 

Algoritmus je implementován v souboru \texttt{get\_tRNA.py} a využívá možností knihovny \texttt{re}. Vstupem programu je soubor se sekvencí genomu ve formátu fasta. Výstupem na standartní výstup je množina záznamů ve formátu multifasta, odpovídající nalezeným tRNA genům.

\begin{algorithm}[H]
\caption{Použitý regulární výraz na hledání tRNA genů}
\label{re}
\begin{algorithmic}[1]
\State $pattern = """$
\State $[AUCG]\{13\}$ \quad \Comment{1 - 13}
\State $A$ \quad \Comment{14}
\State $(A|G)$ \quad \Comment{15}
\State $[AUCG]\{1,3\}$ \quad \Comment{16 - 17A}
\State $G$ \quad \Comment{18}
\State $[AUCG]\{11,14\}$ \quad \Comment{19 - 31}
\State $(A|C|U)$ \quad \Comment{32}
\State $U$ \quad \Comment{33}
\State $(?P<Anticodon>[AUCG]\{3\})$ \quad \Comment{34 - 36 Anticodon}
\State $(A|G)$ \quad \Comment{37}
\State $[AUCG]\{11,31\}$ \quad \Comment{38 - 52}
\State $GUUC$ \quad \Comment{53 - 56}
\State $(G|A)$ \quad \Comment{57}
\State $A$ \quad \Comment{58}
\State $[AUCG]$ \quad \Comment{59}
\State $(U|C)$ \quad \Comment{60}
\State $C$ \quad \Comment{61}
\State $[AUCG]\{12\}$ \quad \Comment{62 - 73}
\State $CCA$ \quad \Comment{74 - 76}
\State $"""$
\end{algorithmic}
\end{algorithm}  

\section{Výsledky porovnání s GTD}
Požadavkem pro úspěšné vyřešení problému je dosáhnutí alespoń 80\% překryvu se záznamy GTD kódujících 20 standartních aminokyselin se skóre alespoń 60. Dalším požadavkem je, že program by neměl vypisovat více než 20\% falešných výskytů (záznamy jež nejsou v GTD). Oba tyto požadavky jsou splněny jak je vidět na výstupu programu:

\begin{verbatim}
$ ./compare_tRNA.py seq/found_tRNA.multifasta seq/known_tRNA.multifasta
tRNA_6|AspCUA|630445|76|+
tRNA_14|AlaCGA|1027722|87|+
tRNA_75|LeuAAC|1812665|90|-
tRNA_76|SerUCA|1742003|83|-
tRNA_77|ArgGCG|1697066|89|-
tRNA_90|SerAGC|408294|93|-

Escherichia_coli_CFT073_chr (236813-236889)  Glu (TTC) 77 bp  Sc: 49.26
Escherichia_coli_CFT073_chr (434982-435062)  Undet (???) 81 bp  Sc: 28.45
Escherichia_coli_CFT073_chr (1211152-1211228)  Arg (TCT) 77 bp  Sc: 59.54
Escherichia_coli_CFT073_chr (1342387-1342463)  Arg (TCG) 77 bp  Sc: 48.60
Escherichia_coli_CFT073_chr (4274308-4274398)  SeC(p) (TCA) 91 bp  Sc: 75.19

neporovnaných get_tRNA.py: 6/90
neporovnaných GTD: 5/89
\end{verbatim}

Program produkuje 7\% falešných výskytů a nenalezne 6\% záznamů v GTD. Všechny nenalezené záznamy obsahují bud' nestandartní aminokyselinu nebo mají skóre nižší jak 60. 

\section{Závěr}
Podařilo se mi naimplementovat algoritmus pro hledání tRNA genů. Program splńuje požadavky ze zadání projektu. Výskyt falešných položek a nenalezení všech záznamů z GTD je způsobeno jednoduchostí použité metody. Falešné výskyty by se mohli redukovat přidáním dodatečné filtrace nalezených záznamů. Pro nalezení většího počtu záznamů z GTD je možno vylepšit stávající algoritmus, případně použít složitější metodu hledání tRNA genů.

\begin{thebibliography}{99}

\bibitem{saks}
  SAKS, Margaret E., CONERY, John S.
  \emph{Anticodon-dependent conservation of bacterial tRNA gene sequences.}
  RNA 13.5 (2007): 651-660.

\end{thebibliography}

\end{document}
